\documentclass[hidelinks, 12pt, a4paper]{article}

\usepackage[margin=0.5in]{geometry}
\usepackage[sfdefault,light]{roboto}
\usepackage[none]{hyphenat}%Remove hyphenation
\usepackage{fontawesome}
\usepackage{hyperref}
\usepackage{tabularx}
\usepackage{tikz}
\usepackage{graphbox}

 \hypersetup{
	colorlinks=true,
	linkcolor=blue,
	filecolor=blue,
	citecolor = black,      
	urlcolor=blue,
}

%No page numbering
\pagenumbering{gobble}
\setlength{\parindent}{0pt}

\begin{document}
	\begin{tabular}{p{0.65\textwidth}p{0.35\textwidth}}
		\begin{minipage}{\linewidth}
			\begin{Huge}Steven Waterman\end{Huge}
			
			\hspace{24pt}\begin{large}Senior Developer at NHS BSA\end{large}\\
			
			I'm a track-proven generalist, able to quickly learn new tech or languages and work across many domains.\\
			
			I can follow best practices, but know I'll need to think outside the box to build reliable solutions when `best practice' is impossible.\\
			
			I'm always happy to spend my time helping others, because in the end it's the users that win --- that's what matters.\\
			
			I'm really ambitious and need to be challenged, flourishing in small teams of brilliant people who will push me to be a better developer.
		\end{minipage} & \vspace{-40pt}\begin{minipage}{\linewidth}
			\begin{flushright}
				\begin{tabular}{rc}
					\href{https://en.wikipedia.org/wiki/Durham,_England}{Durham, UK} & \href{https://en.wikipedia.org/wiki/Durham,_England}{\faHome} \\
					\href{mailto:cv@stevenwaterman.uk}{cv@stevenwaterman.uk} & \href{mailto:cv@stevenwaterman.uk}{\faEnvelope} \\
					\href{http://www.stevenwaterman.uk}{stevenwaterman.uk} & \href{http://www.stevenwaterman.uk}{\faLink} \\
					&\\
					\href{https://www.linkedin.com/in/steven-waterman/}{steven-waterman} & \href{https://www.linkedin.com/in/steven-waterman/}{\faLinkedin} \\
					\href{https://twitter.com/SteWaterman}{SteWaterman} & \href{https://twitter.com/SteWaterman}{\faTwitter} \\
					\href{https://github.com/stevenwaterman}{StevenWaterman} & \href{https://github.com/stevenwaterman}{\faGithub}
				\end{tabular}
			\end{flushright}
		\end{minipage}
	\end{tabular}

	\vspace{12pt}

	\begin{Large}Experience\end{Large}
	\rule{200pt}{1pt}\\
	
	\vspace{-2pt}
	
	\begin{tabularx}{\linewidth}{X r}
		\textbf{Senior Developer @ \href{https://www.nhsbsa.nhs.uk/}{NHS BSA}} & \textbf{Nov 2020 --- Present}
	\end{tabularx}\vspace{2pt}

	\hspace{0.05\linewidth}\begin{minipage}{0.95\linewidth}
		I'm a full stack developer working on the new NHS Jobs, a public-facing Government service.
		That means accessibility is paramount, meeting WCAG AAA and supporting progressive enhancement for users without Javascript.
		We can't release an update with known a11y issues.\\
		
		I perform a wide range of work, from fixing accessibility in CSS to improving performance by tweaking SQL.
		Everything I do is focussed on adding value for users, either directly through my own work or indirectly by helping others. A day could involve observing user research, working with designers and POs to effectively meet user needs, or mentoring junior developers.\\
		
		Between tickets, I have a personal focus on developer experience and tooling.
		Shortly after joining, I set up Docker Compose for local development and onboarded the team, meaning nobody is running 15 microservices manually any more.
		I also refactored our frontend integration tests, creating a custom DSL that makes it easier to do things right.\\
		
		I constantly push for more communication between functions, talking to everyone involved and bridging the gaps between them.
		I ran retrospectives with a focus on action, making them something more than a place to vent and giving the team ownership over our ways of working.\\
	\end{minipage}
	
	
	\begin{tabularx}{\linewidth}{X r}
		\textbf{Developer @ \href{https://www.scottlogic.com/}{Scott Logic}} & \textbf{Aug 2019 --- Oct 2020}
	\end{tabularx}\vspace{2pt}
	
	\hspace{0.05\linewidth}\begin{minipage}{0.95\linewidth}
		As a consultant I worked on a number of demanding projects, often expected to pick up new languages, technologies, or business domains, and be able to contribute within a few days.\\
		
		As the COVID-19 pandemic set in, I worked in a 2/3 person team advising NHS Digital on how to modernise the data pipeline feeding the Shielding Patients List.
		We architected and oversaw the in-place migration from complex SQL queries to a Databricks cluster, ran detailed knowledge transfer sessions, and advised senior leadership on best practices for Data and DevOps.\\
		
		Other projects saw me create inter-service authentication and data auditing functionality,
		transition our ways of working to the new all-remote reality, run the first remote retro, and set up continuous deployment with AWS CDK, including data storage bridging both SQL and NoSQL.\\
	\end{minipage}


	\begin{tabularx}{\linewidth}{X r}
		\textbf{Junior Software Engineer @ \href{https://www.condecoconnect.com/}{Codeco}} & \textbf{Jul 2018 --- Oct 2018}
	\end{tabularx}\vspace{2pt}

	\hspace{0.05\linewidth}\begin{minipage}{0.95\linewidth}
		As an intern working on the backend API, the microservice architecture used at Condeco was completely new to me.
		I spent a few months learning and getting up to speed, culminating in me conceiving, architecting, and pitching a new licencing microservice to technical leadership.
	\end{minipage}
	
	\begin{Large}Education\end{Large}
	\rule{200pt}{1pt}\\
	
	
	\begin{tabularx}{\linewidth}{X r}
		\textbf{\href{https://www.dur.ac.uk/courses/info/?id=11509\&title=Computer+Science\&code=G400\&type=BSC\&year=2016}{BSc Computer Science} @ Durham University (1st Class Hons.)} & \textbf{2016 --- 2019}
	\end{tabularx}\vspace{2pt}

	\hspace{0.05\linewidth}\begin{minipage}{0.95\linewidth}
		Dissertation title: \emph{Tailoring horror games with biosignals}\\
	\end{minipage}

	
	\begin{tabularx}{\linewidth}{X r}
		\textbf{\href{https://www.dur.ac.uk/courses/info/?id=11558\&title=General+Engineering\&code=H100\&type=MENG\&year=2015}{MEng General Engineering} @ Durham University} & \textbf{2015 --- 2016}
	\end{tabularx}\vspace{2pt}

	\hspace{0.05\linewidth}\begin{minipage}{0.95\linewidth}
		I always wanted to work in tech, but an elective CS module convinced me to change course\\
	\end{minipage}

	
	\vspace{8pt}
	\begin{Large}Other Work\end{Large}
	\rule{200pt}{1pt}\\
	
	
	\begin{tabularx}{\linewidth}{X r}
		\textbf{Thought Leadership} & \textbf{Constantly!}
	\end{tabularx}\vspace{2pt}
	
	\hspace{0.05\linewidth}\begin{tabularx}{0.95\linewidth}{Xr}
		\begin{minipage}{\linewidth}
			I'm a seasoned speaker at local tech talks, including that time I threw chocolates at \href{https://www.youtube.com/watch?v=2ibiA5TEsxw}{NE:Tech} and the time I live-coded an underwhelming website with \href{https://svelte.dev}{Svelte} at \href{https://www.youtube.com/watch?v=P6u0Uv_VxCU}{NE-RPC}.
			I've written many \href{https://stevenwaterman.uk/}{blogs}, including my descent into \href{https://blog.scottlogic.com/2020/10/09/ergo-rabbit-hole.html}{ergo-keyboard madness} that went viral and ended up on the \href{https://news.ycombinator.com/item?id=24728224}{HN front page}.
			I've also made regular appearances in \href{https://www.baeldung.com/java-weekly-315}{Java Weekly} with technical blogs like \href{https://blog.scottlogic.com/2020/01/03/rethinking-the-java-dto.html}{Rethinking the Java DTO}.
		\end{minipage} & \href{https://blog.scottlogic.com/2020/10/09/ergo-rabbit-hole.html}{\includegraphics[align=c, width=0.36\textwidth]{keyboard}}
	\end{tabularx}

	\vspace{24pt}
	
	\begin{tabularx}{\linewidth}{X r}
		\textbf{Narration.studio} (\href{https://narration.studio/}{Live}) (\href{https://github.com/stevenwaterman/narration.studio/}{GitHub}) & \textbf{Oct 2020}
	\end{tabularx}\vspace{2pt}

	\hspace{0.05\linewidth}\begin{tabularx}{0.95\linewidth}{Xr}
		\begin{minipage}{\linewidth}
			When I narrated my ergo-keyboard blog post, it was a really tedious manual process.
			That was no good, so I made Narration.studio: an in-browser narration editing tool using the web speech recognition API to be completely hands-free.
			Read the lines of your script as they appear on screen, and redo a line by simply saying it again.
			Narration.studio will detect it and overwrite the previous recording for that line.
			Highlighted in the Dec 2020 \href{https://svelte.dev/blog/whats-new-in-svelte-december-2020}{Svelte Community Showcase}.
		\end{minipage} & \href{https://narration.studio/}{\includegraphics[align=c, width=0.36\textwidth]{narrationstudio}}
	\end{tabularx}
	
	\vspace{24pt}
	
	\begin{tabularx}{\linewidth}{X r}
		\textbf{MuseTree} (\href{https://stevenwaterman.uk/musetree/}{Live}) (\href{https://github.com/stevenwaterman/musetree}{GitHub}) & \textbf{Jan 2020}
	\end{tabularx}\vspace{2pt}
	
	\hspace{0.05\linewidth}\begin{tabularx}{0.95\linewidth}{Xr}
		\begin{minipage}{\linewidth}
			After falling in love with \href{https://openai.com/blog/musenet/}{MuseNet}, OpenAI's MIDI generator, I decided to make a custom front-end for it.
			While the official tool is a simple toy demo, MuseTree is used by creators in real-world scenarios to create songs and jingles.
			As a successful open-source project, it sees frequent contributions from the FOSS community.
			Latest update adds integration with the Web Audio API to perform real-time audio synthesis in-browser.
		\end{minipage} & \href{https://stevenwaterman.uk/musetree/}{\includegraphics[align=c, width=0.36\textwidth]{musetree}}
	\end{tabularx}

	\vspace{24pt}
	
	\begin{tabularx}{\linewidth}{X r}
		\textbf{Sharpshot} (\href{https://github.com/stevenwaterman/sharpshot}{GitHub}) & \textbf{Dec 2018}
	\end{tabularx}\vspace{2pt}
	
	\hspace{0.05\linewidth}\begin{tabularx}{0.95\linewidth}{Xr}
		\begin{minipage}{\linewidth}
				I created the initial version of Sharpshot in 24 hours for \href{http://www.durhack.com}{Durhack 2018}, winning the `GitHub Prize for Best Dev Tool' \& overall runner-up. It's an esoteric visual programming language where nodes are placed on a grid. Each node represents a function, and parameters move around the screen annihilating each other when they collide. An addictive Zachlike puzzle game.
		\end{minipage} & \href{https://github.com/stevenwaterman/sharpshot}{\includegraphics[align=c, width=0.36\textwidth]{sharpshot}}
	\end{tabularx}
\end{document}