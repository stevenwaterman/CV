\documentclass[hidelinks, 12pt, a4paper]{article}

\usepackage[margin=0.5in]{geometry}
\usepackage[sfdefault,light]{roboto}
\usepackage[none]{hyphenat}%Remove hyphenation
\usepackage{fontawesome}
\usepackage{enumitem}
\usepackage{hyperref}
\usepackage{tabularx}
\usepackage{tikz}
\usepackage{graphbox}
\usepackage{multicol}
\usepackage{wrapfig}

 \hypersetup{
	colorlinks=true,
	linkcolor=blue,
	filecolor=blue,
	citecolor = black,      
	urlcolor=blue,
}

%No page numbering
\pagenumbering{gobble}
\setlength{\parindent}{0pt}
\setlength\multicolsep{0pt}

\begin{document}	
	\begin{Huge}Steven Waterman\end{Huge}
	
	\vspace{6pt}\hspace{120pt} \begin{large}Full Stack Developer\end{large}\\
	
	\begin{wraptable}{r}{5cm}
		\vspace{-75pt}
		\begin{tabular}{rc}
			\href{https://en.wikipedia.org/wiki/Durham,_England}{Durham, UK} & \href{https://en.wikipedia.org/wiki/Durham,_England}{\faHome} \\
			\href{mailto:cv@stevenwaterman.uk}{cv@stevenwaterman.uk} & \href{mailto:cv@stevenwaterman.uk}{\faEnvelope} \\
			\href{http://www.stevenwaterman.uk}{stevenwaterman.uk} & \href{http://www.stevenwaterman.uk}{\faLink} \\
			&\\
			\href{https://www.linkedin.com/in/steven-waterman/}{steven-waterman} & \href{https://www.linkedin.com/in/steven-waterman/}{\faLinkedin} \\
			\href{https://twitter.com/SteWaterman}{SteWaterman} & \href{https://twitter.com/SteWaterman}{\faTwitter} \\
			\href{https://github.com/stevenwaterman}{StevenWaterman} & \href{https://github.com/stevenwaterman}{\faGithub}
		\end{tabular}
	\end{wraptable}
	
	I'm a generalist Software Engineer with a love of learning new things and transferring knowledge between domains.
	I have a strong interest in the team dynamics that underlie development, always looking to find the root cause of any technical problems and work within the team to fix them.
	
	\vspace{40pt}
	\begin{Large}Tech Stacks\end{Large}\\

	I'm always learning new things, so this list is quite long.
	The first line of each section is my first choice, and what I'd go for by default.
	After that, tech is listed from most-least comfortable.\\
	
	\begin{multicols}{2}
		\textbf{Styling}: Pure CSS (\href{https://stevenwaterman.uk/you-dont-need-js/}{lots of it}) + SVGs\\
		SCSS, GOV.UK Frontend, Bootstrap, Bulma\\
		
		\textbf{Frontend}: SvelteKit\\
		Svelte, React (Redux), Express, Spring\\
		jQuery, ASP.NET, Jekyll, Next.js\\
		
		\textbf{Backend}: Express\\
		SvelteKit, Spring (Boot), ASP.NET\\
		
		\textbf{Data}: Firestore, Postgres\\
		Other SQL, MonogDB, DynamoDB, DataBricks\\
		Apache Spark, Redis, GraphQL\\
		
		\textbf{Frontend Hosting}: GitHub Pages\\
		GCP Cloud Storage, AWS S3, IBM\\
		
		\textbf{Backend Hosting}: Serverless (GCP / AWS)\\
		Docker, GCP App Engine, None (SPA + Firestore)\\
		
		\textbf{CI}: GitHub Actions\\
		GitLab CI, CircleCI\\
		
		\textbf{CD}: Terraform\\
		AWS CDK, AWS CloudFormation\\
		
		\textbf{Cloud Providers}: GCP\\
		Firebase, AWS, DigitalOcean\\
		Azure, IBM Cloud\\
		
		\textbf{Languages}: TypeScript, Kotlin\\
		JavaScript, Java, Python, C\#, Bash\\
		Lua, C, R, Clojure, MatLab, VB, Haskell, Prolog\\
	\end{multicols}

	\vspace{40pt}
	\begin{Large}Style\end{Large}\\
	
	\begin{multicols}{2}
		As a developer, I'm very people-oriented.
		Code is written by people, meaning many technical problems are actually caused by a lack of communication, trust, or alignment on purpose.
		Issues in the code are inevitable unless we resolve the conflict within the team first.\\
		
		I pay extra attention to the developer experience when writing code, spending the extra time to think through API design and how to implement semantic type constraints that prevent misunderstandings at compile-time.\\
		
		As a generalist, my most important skill is the ability to learn quickly by relying on my past experiences.
		I can see similarities between tasks and transfer knowledge from one area to another.
		I won't be an expert, but I'll know enough to work with experts.\\
		
		I always prioritise expressing myself clearly.
		My tech blog has a small following, with a few viral hits.
		A history in improvised comedy means I'm a calm and adaptable presenter, often giving tech talks or live-coding on Twitch.\\
	\end{multicols}
	
	\newpage
	
	\begin{tabularx}{\textwidth}{@{}llXrr@{}}
		\begin{Large}Career History\end{Large}&
		\rule{80pt}{1pt}&&&
	\end{tabularx}\\
	
	\begin{tabularx}{\linewidth}{@{}Xr@{}}
		\textbf{Founder @ \href{https://www.lexoral.com/}{Lexoral}} & \textbf{May 2021 --- Present}
	\end{tabularx}\vspace{2pt}

	\begin{multicols}{2}
		After watching my partner struggle to transcribe their PhD interviews, I founded Lexoral.
		We give you an AI \emph{assistant} that transcribes the easy bits for you, asking for help when it's not sure.\\
		
		I ran all aspects of the business, from design to development, accounting to marketing.
		Lexoral closed out 2021 by joining the \href{https://dcincubator.co.uk/}{Durham City Incubator}, an intensive 6-month program.\\
		
		I've discovered how much I didn't know before, getting real experience interviewing users, designing a marketing strategy, and building a meaningful value prop based on the things customers actually cared about.\\
		
		After hearing concerns about data security, I embraced a philosophy of radical transparency.
		To prove that we weren't hiding anything, Lexoral went \href{https://github.com/stevenwaterman/Lexoral/}{open-source} with public CD pipelines, and most of it was written live on \href{https://twitch.tv/lexoral}{Twitch}.\\
		
		Lexoral was a crash-course in cloud-native development.
		Everything is serverless, clients talk to Firestore \href{https://lexoral.com/blog/svelte-firestore-binding/}{directly}, and \href{https://twitter.com/SteWaterman/status/1445041856023339011}{one part of the pipeline} runs on over 1000 instances in parallel (per user!).\\
		
		I still believe that Lexoral can be successful.
		It has a niche, and solves a real need.
		However, working solo is not for me, and I'm missing the people-oriented aspects of larger teams.\\
		
	\end{multicols}


	\begin{tabularx}{\linewidth}{@{}Xr@{}}
		\textbf{Senior Developer @ \href{https://www.nhsbsa.nhs.uk/}{NHS BSA}} & \textbf{Nov 2020 --- May 2021}
	\end{tabularx}\vspace{2pt}
	
	\begin{multicols}{2}
		Shocked at the poor developer experience caused by outdated tooling and organisational barriers, I made it my mission to resolve those issues.
		Seeing people run 15 microservices by hand, I moved development to containers and onboarded the team.
		After spending half an hour setting up test data, I refactored our frontend integration tests, creating a custom DSL that makes it trivial.\\
		
		I constantly pushed for more communication between functions and ran action-focussed retros, making it more than just a place to vent and giving the team ownership over our ways of working.
		Since leaving, they have adopted some of my more radical ideas, including a complete restructuring of the project to de-silo the teams and allowing people to self-organise.\\
	\end{multicols}

	\begin{tabularx}{\linewidth}{@{}Xr@{}}
		\textbf{Consultant Developer @ \href{https://www.scottlogic.com/}{Scott Logic}} & \textbf{Aug 2019 --- Oct 2020}
	\end{tabularx}\vspace{2pt}
	
	\begin{multicols}{2}
		I worked on a number of demanding projects, often expected to pick up new languages, technologies, or business domains, and be able to contribute within a few days.\\
		
		As the COVID-19 pandemic set in, I worked in a trio advising NHS Digital on how to rearchitect the data pipeline feeding the Shielding Patients List.
		We planned and oversaw the in-place migration from complex SQL queries to a Databricks cluster, ran detailed knowledge transfer sessions with NHS devs, and advised senior leadership on how to prevent similar situations in future.\\
		
		Prior to that, I worked on an upcoming product for a multinational bank, and a Twitter-like equities research platform.
		I have always been drawn to developer-facing improvements, working on projects like inter-service authentication, data auditing, and continuous deployment pipelines.\\
		
		Throughout my time at Scott Logic, I actively pushed to improve our ways of working.
		I ran retros and knowledge-sharing sessions to help integrate with client development teams.
		To make sure there was a lasting record to learn from, I documented the decisions we made in a wiki.\\
	\end{multicols}

	\vspace{24pt}
	\begin{tabularx}{\textwidth}{@{}llXrr@{}}
		\begin{Large}Education\end{Large}&
		\rule{80pt}{1pt}&&&
	\end{tabularx}\\
	
	
	\begin{tabularx}{\linewidth}{@{}Xr@{}}
		\textbf{\href{https://www.dur.ac.uk/courses/info/?id=11509\&title=Computer+Science\&code=G400\&type=BSC\&year=2016}{BSc Computer Science} @ Durham University (1st Class Hons.)} & \textbf{2016 --- 2019}
	\end{tabularx}\vspace{2pt}

	Dissertation title: \emph{Tailoring horror games with biosignals}\\

	\begin{tabularx}{\linewidth}{@{}Xr@{}}
		\textbf{\href{https://www.dur.ac.uk/courses/info/?id=11558\&title=General+Engineering\&code=H100\&type=MENG\&year=2015}{MEng General Engineering} @ Durham University (Certificate)} & \textbf{2015 --- 2016}
	\end{tabularx}\vspace{2pt}

	Changed course after taking an elective CS module.
	A year of Engineering is surprisingly handy!
	
	\newpage
	
	\begin{tabularx}{\textwidth}{@{}llXrr@{}}
		\begin{Large}Other Work\end{Large}&
		\rule{80pt}{1pt}&&
		\rule{63pt}{1pt}&
		\textbf{Writing}
	\end{tabularx}\\

	I've written for many tech blogs over the years. Here are a few of my highlights:

	\begin{itemize}
		\item \begin{small}\href{https://lexoral.com/blog/you-dont-need-js/}{\textbf{5 things you don't need Javascript for}} - A tour of some lesser-known HTML and CSS features that let you create sleek websites without JS, from animated diagrams to dark mode.\end{small}
		
		\item \begin{small}\href{https://lexoral.com/blog/svelte-firestore-binding/}{\textbf{Database sync like magic, with Svelte + Firestore}} - Discussing Lexoral's data layer, built from first principles.\end{small}
		
		\item \begin{small}\href{https://blog.scottlogic.com/2020/10/09/ergo-rabbit-hole.html}{\textbf{Down the ergonomic keyboard rabbit hole}} - The story of how I ended up with \href{https://blog.scottlogic.com/swaterman/assets/ergo-rabbit-hole/layer0.png}{such} a weird \href{https://ergodox-ez.com/}{keyboard}.\end{small}
		
		\item \begin{small}\href{https://blog.scottlogic.com/2020/01/03/rethinking-the-java-dto.html}{\textbf{Rethinking the Java DTO}} - Exploring how we added extra type constraints onto our DTOs using Lombok, making them more flexible and more resistant to runtime errors. Featured in \href{https://www.baeldung.com/java-weekly-315}{Java Weekly}.\end{small}
	\end{itemize}
	
	\begin{tabularx}{\textwidth}{@{}Xrr@{}}&
		\rule{50pt}{1pt}&
		\textbf{Speaking}
	\end{tabularx}\\

	I'm a seasoned speaker, and have given a number of tech talks at local meetups, including:
	
	\begin{itemize}
		\item \begin{small}\href{https://www.youtube.com/watch?v=2ibiA5TEsxw}{\textbf{NE:Tech}} - Where I talked about Minesweeper and threw chocolates at people.\end{small}
		
		\item \begin{small}\href{https://www.youtube.com/watch?v=P6u0Uv_VxCU}{\textbf{NE-RPC}} - Where I live-coded a website from scratch in Svelte.\end{small}
		
		\item \begin{small}\href{https://www.twitch.tv/stewaterman}{\textbf{Twitch}} - Where I regularly live-code my projects (mixed with some games).\end{small}
	\end{itemize}


	
	\begin{tabularx}{\textwidth}{@{}Xrr@{}}&
		\rule{50pt}{1pt}&
		\textbf{Projects}
	\end{tabularx}\\

	I'm always working on something new, take a look at some of my side projects:
	
	\begin{itemize}
		\item \begin{small}\href{https://github.com/stevenwaterman/narration.studio}{\textbf{Narration.Studio}} - an in-browser narration editing tool using the web speech recognition API to be completely hands-free. Integrates with WebGL for high-performance waveform rendering. As a result of being built entirely with pre-release APIs, it no longer works. Looking back, this is the precursor to Lexoral.\end{small}
		
		\item \begin{small} \href{https://stevenwaterman.uk/musetree/}{\textbf{MuseTree}} (\href{https://github.com/stevenwaterman/musetree}{Source}) - A custom tree-based frontend for OpenAI's \href{https://openai.com/blog/musenet/}{MuseNet}, made for real music production workloads. I hand-wrote the instrument synthesisers you hear using the Web Audio API.\end{small}
		
		\item \begin{small}\href{https://stevenwaterman.uk/NoTimeToStalk}{\textbf{No Time To Stalk}} (\href{https://github.com/stevenwaterman/NoTimeToStalk}{Source}) - an experimental murder mystery game that was secretly multiplayer. Every action you take is recorded, and you become an NPC for the next player. How long until \emph{you} get accused?\end{small}
		
		\item \begin{small}\href{https://github.com/stevenwaterman/sharpshot}{\textbf{Sharpshot}} - an esoteric visual programming language where data flies around a 2d grid and can collide in mid-air. Created for \href{http://www.durhack.com}{Durhack 2018}, winning the \emph{GitHub Prize for Best Dev Tool} and 2\textsuperscript{nd} place overall.\end{small}
		
		\item \begin{small}\href{https://linktr.ee/prevoid_art}{\textbf{Prevoid}} - An exploration into AI-generated art using CLIP-guided diffusion models.\end{small}
		
		\item \begin{small}\href{https://soundcloud.com/user-872603169/welcome-to-the-theatre}{\textbf{Soundcloud}} - A few samples of my music and remixes I've created.\end{small}
	\end{itemize}


	\begin{tabularx}{\textwidth}{@{}Xrr@{}}&
		\rule{50pt}{1pt}&
		\textbf{Weird}
	\end{tabularx}\\

	It's always good to have a few fun facts to hand, so here are some of mine:
	
	\begin{itemize}
		\item \begin{small}I performed improvised comedy at the Edinburgh Fringe Festival.\end{small}
		
		\item \begin{small}I built an electric bike that can do 50mph.\end{small}
		
		\item \begin{small}I helped design a Robot Wars bot.\end{small}
		
		\item \begin{small}I founded the Durham University Bureaucracy Society, with the aim of growing to the point that we were too bureacratic to have any spare time for recruitment. It only took 12 members.\end{small}
	\end{itemize}
\end{document}